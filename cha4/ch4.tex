\documentclass[UTF8,a4paper,10pt]{ctexart}
\usepackage[left=2.50cm, right=2.50cm, top=2.50cm, bottom=2.50cm]{geometry}

% -- text font --
% compile using Xelatex

%\setmainfont{Microsoft YaHei}  % 微软雅黑
%\setmainfont{YouYuan}  % 幼圆    
%\setmainfont{NSimSun}  % 新宋体
%\setmainfont{KaiTi}    % 楷体
%\setmainfont{SimSun}   % 宋体
%\setmainfont{SimHei}   % 黑体

\usepackage{times}
\usepackage{diagbox}
%\usepackage{mathpazo}
%\usepackage{fourier}
%\usepackage{charter}
%\usepackage{helvet}

\usepackage{amsmath, amsfonts, amssymb} % math equations, symbols
\usepackage[english]{babel}
\usepackage{color}      % color content
\usepackage{graphicx}   % import figures
\usepackage{url}        % hyperlinks
\usepackage{bm}         % bold type for equations
\usepackage{multirow}
\usepackage{booktabs}
\usepackage{epstopdf}
\usepackage{epsfig}
\usepackage{algorithm}
\usepackage{algorithmic}
%\renewcommand{\algorithmicrequire}{ \textbf{Input:}}     % use Input in the format of Algorithm  
%\renewcommand{\algorithmicensure}{ \textbf{Initialize:}} % use Initialize in the format of Algorithm  
%\renewcommand{\algorithmicreturn}{ \textbf{Output:}}     % use Output in the format of Algorithm  

\def\deriv{d\,}
\usepackage{fancyhdr}   % 设置页眉、页脚
\pagestyle{fancy}
\lhead{2019概率论统计分析及量测技术作业答案}
\chead{}
\rhead{第6周}
\cfoot{\thepage}
   

\usepackage{hyperref}   % bookmarks
\hypersetup{colorlinks, bookmarks, unicode} % unicode


\title{第四章作业}
%\author{ 作者 Author \thanks{作者介绍 Brief introduction} }
%\date{\today}

\begin{document}



\section{课本习题}

\subsection{P114/7}
\noindent(1)

(i)
\begin{align}
    E(Y) &= E(2X) = \int_{-\infty}^\infty 2xf(x)\,\text{d}x \nonumber \\
         &= 2 \int_0^\infty x e^{-x}\,\text{d}x = 2;
\end{align}

(ii)
\begin{equation}
    E(Y) = E(e^{-2X}) = \int_0^\infty e^{-2x}\cdot e^{-x}\,\text{d}x = {1\over3}.\\
\end{equation}

\noindent(2)

$X_i$的分布函数为
\begin{equation}
    F(x)= \left\{
    \begin{array}{lr}
    0,& x<0, \\
    x, & 0\leq x<1, \\
    1, & x\geq1 . 
    \end{array}
    \right.
\end{equation}

(i)

$X_1, X_2, \cdots, X_n$ 独立,所以$U=\max\{X_1,X_2,\cdots,X_n\}$的分布函数为
\begin{equation}
    F_U(u)= \left\{
    \begin{array}{lr}
    0,& u<0, \\
    u^n, & 0\leq u<1, \\
    1, & u\geq1 . 
    \end{array}
    \right.
\end{equation}

$U$的概率密度函数为
\begin{equation}
    f_U(u)= \left\{
    \begin{array}{lr}
        nu^{n-1}, & 0<u<1, \\
        0, & \text{其他}.
    \end{array}
    \right.
\end{equation}

\begin{equation}
    E(U) = \int_0^1 u \cdot nu^{n-1}\,\text{d}u = {n\over n+1}.
\end{equation}

(ii)


$V=\min\{X_1,X_2,\cdots,X_n\}$的分布函数为
\begin{equation}
    F_V(v)= \left\{
    \begin{array}{lr}
    0,& v<0, \\
    1-(1-v)^n, & 0\leq v<1, \\
    1, & v\geq1 . 
    \end{array}
    \right.
\end{equation}

$V$的概率密度函数为
\begin{equation}
    f_V(v)= \left\{
    \begin{array}{lr}
        n(1-v)^{n-1}, & 0<v<1, \\
        0, & \text{其他}.
    \end{array}
    \right.
\end{equation}

\begin{equation}
    E(V) = \int_0^1 v \cdot n(1-v)^{n-1}\,\text{d}v = {1\over n+1}.
\end{equation}

\subsection{P115/10}
\noindent(1)

由对称性知
\begin{equation}
    E({X^2\over X^2+Y^2}) = E({Y^2\over X^2+Y^2}).
\end{equation}

而
\begin{equation}
    E({X^2\over X^2+Y^2}) + E({Y^2\over X^2+Y^2}) = E(1) = 1.
\end{equation}

所以
\begin{equation}
    E({X^2\over X^2+Y^2}) = E({Y^2\over X^2+Y^2}) = {1\over2}.
\end{equation}

\noindent(2)

$X$与$Y$独立,因而$(X,Y)$的联合概率密度函数为
\begin{equation}
    f(x,y) = {1\over 2\pi\sigma^2}e^{-{x^2+y^2\over 2\sigma^2}} .
\end{equation}

原点到$(X,Y)$的距离为$R = \sqrt{X^2+Y^2}$,期望为
\begin{equation}
    E(R) = \int_{-\infty}^\infty\int_{-\infty}^\infty \sqrt{X^2+Y^2} {1\over 2\pi\sigma^2}e^{-{x^2+y^2\over 2\sigma^2}} \, \text{d}x\text{d}y.
\end{equation}

换成极坐标系
\begin{align}
    E(R) &= \int_0^{2\pi}\text{d}\theta \int_0^\infty {r\over 2\pi \sigma^2}e^{-{r^2\over 2\sigma^2}}\,r\text{d}r = \sigma \sqrt{\pi\over2}.
\end{align}


\subsection{P115/12}
\begin{align}
    E(S) = E\left(\frac{\pi d^2}{4}\right) &= \int_a^b \frac{\pi d^2}{4}\frac{1}{b - a} \text{d} d \nonumber \\
                                           &= \frac{\pi(a^2 + ab + b^2)}{12}.
\end{align}

\subsection{P115/15}
设
\begin{equation}
    X_i = \left\{
    \begin{array}{lr}
    1,& \mbox{第$i$个球放入了第$i$号盒子}, \\
    0, & \mbox{第$i$个球未放入第$i$号盒子},
    \end{array}
    \right.
\end{equation}
则$X_i$满足0-1分布,$P\{X_i = 1\} = \displaystyle\frac{1}{n}$,$P\{X_i = 0\} = 1-\displaystyle\frac{1}{n}$,因此
\begin{equation}
    E(X) = \sum_i E(X_i) = n\cdot\frac{1}{n} = 1.
\end{equation}

\subsection{P116/20}
\begin{align}
    E(X) &= \sum_{k = 1}^{+\infty}kp(1-p)^{k-1} \nonumber \\
         &= p\sum_{k = 1}^{+\infty}k(1-p)^{k-1} \nonumber \\
         &= p\sum_{k = 1}^{+\infty}\frac{\text{d}}{\text{d} (1 - p)}(1 - p)^k \nonumber \\
         &= p\frac{\text{d}}{\text{d} (1 - p)}\left(\sum_{k = 1}^{+\infty}(1 - p)^k\right) \nonumber \\
         &= p\cdot\frac{1}{p^2} = \frac{1}{p},\\
    E(X(X + 1)) &= \sum_{k = 1}^{+\infty}k(k + 1)p(1-p)^{k-1} \nonumber \\
                &= p\sum_{k = 1}^{+\infty}\frac{\text{d}^2}{\text{d} (1 - p)^2}(1 - p)^{k+1} \nonumber \\
                &= p\frac{\text{d}^2}{\text{d} (1 - p)^2}\left(\sum_{k = 1}^{+\infty}(1 - p)^{k+1}\right) \nonumber \\
                &= \frac{2}{p^2}, \\
    D(X) &= E(X^2) - [E(X)]^2 \nonumber \\
         &= E(X(X + 1)) - E(X) - [E(X)]^2 \nonumber \\
         &= \frac{2}{p^2} - \frac{1}{p} - \frac{1}{p^2} \nonumber \\
         &= \frac{1 - p}{p^2}.
\end{align}




\subsection{P117/28}
\begin{align}
    f_X(x) =& \int_{-\sqrt{1 - x^2}}^{+\sqrt{1 - x^2}}f(x, y)\deriv y \nonumber \\
           =& \frac{2}{\pi}\sqrt{1 - x^2}, \\
    f_Y(y) =& \int_{-\sqrt{1 - y^2}}^{+\sqrt{1 - y^2}}f(x, y)\deriv x \nonumber \\
           =& \frac{2}{\pi}\sqrt{1 - y^2}, \\
    E[X] =& \int_{-1}^{1}xf_X(x)\deriv x \nonumber \\
         =& 0, \\
    E[Y] =& \int_{-1}^{1}yf_Y(y)\deriv y \nonumber \\
         =& 0,\\
    E[XY] =& \int_{-1}^{1}\deriv x\int_{-\sqrt{1 - x^2}}^{+\sqrt{1 - x^2}}xyf(x, y)\deriv y \nonumber \\
          =& 0,\\
\end{align}
因此$E[XY] = E[X]E[Y]$,则$\mathrm{cov}(X,Y) = 0, \rho = 0$,$X$与$Y$不相关。但显然$f_X(x)f_Y(y) \not= f(x,y)$,因此$X$与$Y$不独立。


\subsection{P117/32}
\begin{align}
    E[X] &= \int_{0}^2\deriv x\int_0^2xf(x,y)\deriv y \nonumber \\
         &= \frac{7}{6}, \\
    E[Y] &= \int_{0}^2\deriv x\int_0^2yf(x,y)\deriv y \nonumber \\
         &= \frac{7}{6}, \\
    E[XY] &= \int_{0}^2\deriv x\int_0^2xyf(x,y)\deriv y \nonumber \\
          &= \frac{4}{3}, \\
    \mathrm{cov}(X,Y) &= E[XY] - E[X]E[Y] = -\frac{1}{36}, \\
    E[X^2] &= \int_{0}^2\deriv x\int_0^2x^2f(x,y)\deriv y \nonumber \\
           &= \frac{5}{3}, \\
    E[Y^2] &= \int_{0}^2\deriv x\int_0^2y^2f(x,y)\deriv y \nonumber \\
           &= \frac{5}{3}, \\
    D[X] &= E[X^2] - (E[X])^2 = \frac{11}{36}, \\
    D[Y] &= E[Y^2] - (E[Y])^2 = \frac{11}{36}, \\
    \rho &= \frac{\mathrm{cov}(X, Y)}{\sqrt{D[X]}\sqrt{D[Y]}} = -\frac{1}{11}, \\
    D[X + Y] &= D[X] + D[Y] + 2\mathrm{cov}(X, Y) = \frac{5}{9}.
\end{align}

\subsection{P117/33}
由于$X$,$Y$相互独立,$\rho_{XY} = 0$,则
\begin{align}
    E[Z_1] &= \alpha E[X] + \beta E[Y] = (\alpha + \beta)\mu, \\
    E[Z_2] &= \alpha E[X] - \beta E[Y] = (\alpha - \beta)\mu, \\
    E[Z_1Z_2] &= E[\alpha^2X^2 - \beta^2 Y^2] = (\alpha^2 - \beta^2)(\mu^2 + \sigma^2), \\
    D[Z_1] &= \alpha^2 D[X] + \beta^2 D[Y] = (\alpha^2 + \beta^2)\sigma^2, \\
    D[Z_2] &= \alpha^2 D[X] + \beta^2 D[Y] = (\alpha^2 + \beta^2)\sigma^2,
\end{align}
因此
\begin{equation}
    \rho_{Z_1Z_2} = \frac{E[Z_1Z_2] - E[Z_1]E[Z_2]}{\sqrt{D[Z_1]}\sqrt{D[Z_2]}} =\frac{\alpha^2 - \beta^2}{\alpha^2 + \beta^2}.
\end{equation}

\subsection{P117/34}
(1)
\begin{align}
    E[X^2] &= D[X] + (E[X])^2 = 4, \\
    E[Y^2] &= D[Y] + (E[Y])^2 = 16, \\
    E[XY] &= \rho_{XY}\sqrt{D[X]}\sqrt{D[Y]} + E[X]E[Y] = -4,
\end{align}
因此
\begin{align}
    E[W] &= E[a^2X^2 + 6aXY + 9Y^2] \nonumber \\
         &= a^2E[X^2] + 6aE[XY] + 9E[Y^2] \nonumber \\
         &= 4(a^2 - 6a + 36),
\end{align}
对上式求极值,得到:当$a = 3$时,$E[W]$取最小值108。

(2)
\begin{align}
    \mathrm{cov}(W,V) &= E[WV] - E[W]E[V]  \nonumber \\
                 &= E[X^2 - a^2Y^2] - E[X + aY]E[X - aY] \nonumber \\
                 &= E[X^2] - a^2E[Y^2] - (E[X] + aE[Y])(E[X] - aE[Y]) \nonumber \\
                 &= \sigma_X^2 + \mu_X^2 - a^2\sigma_Y^2 - a^2\mu_Y^2 - (\mu_X^2 - a^2\mu_Y^2) \nonumber \\
                 &= \sigma_X^2 - a^2\sigma_Y^2,
\end{align}
当$a^2 = \displaystyle\frac{\sigma_X^2}{\sigma_Y^2}$时,$\mathrm{cov}(W,V)=0$,即$\rho_{WV} = 0$。同时由于$W,V$是服从二维正态分布的随机变量的线性组合,因此$W,V$也服从二维正态分布。其相关系数$\rho = 0$说明它们相互独立。


\subsection{P117/37}
(1)若$E(V^2)=0$,则$D(V)=0$,从而$P\{V=0\}=1$, $E(VW)=0$,不等式取等号。

$E(W^2)=0$时,同理。

(2)若$E(V^2)>0, E(W^2)>0$,考虑实变量$t$的函数:
\begin{equation}
    q(t)=E[(V+tW)^2] = E(V^2) +2tE(VW)+t^2E(W^2).
\end{equation}
对于任意$t$,$q(t)\geq0$, $E(W^2)>0$,所以二次三项式$q(t)$的判别式
\begin{equation}
    \Delta = 4[E(VW)]^2-4E(V^2)E(W^2)\leq0,
\end{equation}

即
\begin{equation}
    [E(VW)]^2 \leq E(V^2) E(W^2).
\end{equation}

取等条件为$\Delta=0$, 即$q(t)=0$有一个根。此时$t= -E(VW)/E(W^2)$。
\begin{equation}
    V = -t W =  {E(VW)\over E(W^2)} W.
\end{equation}

即当$V=c W$ 时,等号成立($c$为任一常数)。

\section{补充题}

\subsection{}
设$X$为三个人中生日在第一季度的人数,$X = 0, 1, 2, 3$,则$X\sim b(3, 0.25)$,因此$X$的分布律为
\begin{center}
    \begin{tabular}{c|cccc}
        $X$ & 0 & 1 & 2 & 3 \\
        \hline
        $p_k$ & $\frac{27}{64}$ & $\frac{27}{64}$ & $\frac{9}{64}$ & $\frac{1}{64}$
    \end{tabular}
\end{center}

因此$X$的数学期望为
\begin{equation}
    E(X) = \sum_{k}kp_k = 0.75.
\end{equation}

\subsection{}
(1) 
\begin{align}
    E\left(\frac{1}{x}\right) &= \int_{\alpha}^{\beta} \frac{1}{x}\frac{1}{\beta - \alpha} \text{d} x \nonumber \\
                              &= \frac{\ln(\beta/\alpha)}{\beta - \alpha}.
\end{align}

(2)
\begin{align}
    1 / E(x) &= 1 / \int_{\alpha}^{\beta} x \frac{1}{\beta - \alpha} \text{d} x \nonumber \\
             &= \frac{2}{\alpha + \beta}.
\end{align}

\subsection{}
(1)
\begin{align}
    a + 0.2 + b + 0.1 &= 1, \\
    E(x) = -a + 0.2 + 2b + 0.3 &= 1,
\end{align}
解得
\begin{equation}
    a = 0.3,~b = 0.4.
\end{equation}

(2) 随机变量$Y = \displaystyle\frac{1}{X}$的分布律为
\begin{center}
    \begin{tabular}{c|cccc}
        $Y = \frac{1}{X}$ & -1 & $\frac{1}{3}$ & $\frac{1}{2}$ & 1 \\
        \hline
        $p_k$ & 0.3 & 0.1 & 0.4 & 0.2 
    \end{tabular}
\end{center}

则$E(Y) = \displaystyle\sum_{k}kp_k = \displaystyle\frac{2}{15}$,
\begin{align}
    D(Y) &= \sum_k (k - E(Y))^2p_k \nonumber \\
         &= 0.593.
\end{align}

\subsection{}
由柯西-施瓦茨不等式
\begin{equation}
    E({1\over X_2}) E(X_2) = E(({1\over \sqrt{X_2}})^2) E((\sqrt{X_2})^2)\geq E(1)^2 = 1.
\end{equation}

等号成立当且仅当存在常数,$\sqrt{X_2} = c/\sqrt{X_2}$,即$X_2=c$时成立。$X_1,X_2$独立,则$X_1, 1/X_2$也独立,所以
\begin{equation}
    E(X_1/X_2) = E(X_1)E(1/X_2) \geq E(X_1)/E(X_2) = 1.
\end{equation}

等号成立当且仅当$X_1, X_2$只取一个常数$c$时成立。

\end{document}