\documentclass[10pt,a4paper]{article}
\usepackage{fontspec}
\defaultfontfeatures{Mapping=tex-text}
\usepackage{xunicode}
\usepackage{xltxtra}
\usepackage{xeCJK}
\usepackage{ctex}
\usepackage{polyglossia}
\setdefaultlanguage{english}
\usepackage{indentfirst}
\setlength{\parindent}{2em}%中文缩进两个汉字位
\usepackage{amsmath}
\usepackage{amsfonts}
\usepackage{amssymb}
\usepackage{siunitx}
\usepackage{float}
\usepackage{varioref}
\usepackage{cite}
\usepackage[colorlinks,linkcolor=black,anchorcolor=black,citecolor=black]{hyperref}%设置链接的颜色均为黑色
\usepackage{amsmath}
 \usepackage{siunitx}
\usepackage{geometry}

\begin{document}\
\textbf{切比雪夫不等式:}设随机变量X的数学期望和方差都存在,则对任意常数 $\varepsilon>0$,有
$$
\mathrm{P}(|X-E X| \geq \varepsilon) \leq \frac{\operatorname{Var}(X)}{\varepsilon^{2}} \quad \text { 或 } \quad \mathrm{P}(|X-E X| \geq \varepsilon) \geq 1-\frac{\operatorname{Var}(X)}{\varepsilon^{2}}
$$
存在 $\varepsilon_{0}>0$ 使得等号成立的充要条件为 $P\left(X=E X-\varepsilon_{0}\right)=\frac{1-p}{2}, P\left(X=E X+\varepsilon_{0}\right)=\frac{1-p}{2}$, 其中 $p=P(X=E X)$.


\textbf{证明:}


$\mathrm{I}$ 、 充分性: 如果随机变量满足:

\begin{flushleft}
$$P\left(X=E X-\varepsilon_{0}\right)=\frac{1-p}{2}$$
$$P(X=E X)=p$$
$$P\left(X=E X+\varepsilon_{0}\right)=\frac{1-p}{2}$$ 则:
$$
P\left(|X-E X| \geq \varepsilon_{0}\right)=P\left(|X-E X|=\varepsilon_{0}\right)=P\left(X=E X+\varepsilon_{0}\right)+P\left(X=E X-\varepsilon_{0}\right)=1-p
$$
$$\operatorname{Var}(X)=E(X-E X)^{2}=\varepsilon_{0}^{2} \frac{1-p}{2}+\left(-\varepsilon_{0}\right)^{2} \frac{1-p}{2}+0^{2} p=(1-p) \varepsilon_{0}^{2}$$
 由此可得:
$$
\mathrm{P}\left(|X-E X| \geq \varepsilon_{0}\right)=\frac{\operatorname{Var}(X)}{\varepsilon_{0}^{2}}
$$
\end{flushleft}
II 、必要性: 设随机变量 $\mathrm{X}$ 的分布函数为 $\mathrm{F}_{\mathrm{X}}(\mathrm{x})$

由题设可知 
$$\varepsilon_{0}^{2} P\left(|X-E X| \geq \varepsilon_{0}\right)=\operatorname{Var}(X)$$
而
$$\operatorname{Var}(X) = \int_{|x-E X |< \varepsilon_{0}}(x-E X)^{2} d F_{X}(x)+\int_{|x-E X |\geq \varepsilon_{0}}(x-E X)^{2} d F_{X}(x)$$
假设 $P\left(0<| X-E X | <\varepsilon_{0}\right)>0$ 则 :
$$\int_{0<| x-E X|<\varepsilon_{0}}(x-E X)^{2} d F_{X}(x)>0$$ 
于是有:
$$\operatorname{Var}(X) \geq \int_{| x-E X |< \varepsilon_{0}}(x-E X)^{2} d F_{X}(x)+\varepsilon_{0}^{2} P\left(|X-E X |\geq \varepsilon_{0}\right)>\varepsilon_{0}^{2} P\left(|X-E X| \geq \varepsilon_{0}\right)$$
 与题设矛盾, 故 $P\left(0<|X-E X |< \varepsilon_{0}\right)=0$, 由前面证明可知
$$
\operatorname{Var}(X)=\int_{|x-E X| \geq \varepsilon_{0}}(x-E X)^{2} d F_{X}(x)
$$
假设 $P=\left(|X-E X|>\varepsilon_{0}\right)>0$,则得:$$\int_{|x-E X| > \varepsilon_{0}}(x-E X)^{2} d F_{x}(x)>0$$
于是有 
$$
\operatorname{Var}(X)=\varepsilon_{0}^{2} P\left(|X-E X|=\varepsilon_{0}\right)+\int_{|x-E X|>\varepsilon_0}(x-E X)^{2} d F_{X}(x)>\varepsilon_{0}^{2} P\left(|X-E X|=\varepsilon_{0}\right)
$$
这与题设矛盾, 故 $P=\left(|X-E X|>\varepsilon_{0}\right)=0$, 于是得到:
$$
\begin{aligned}
&P=\left(|X-E X|=\varepsilon_{0}\right)=1-P(|X-E X|=0)=1-p \\ \text { 即 } \\
&P\left(X=E X-\varepsilon_{0}\right)=\frac{1-p}{2}\\
&P\left(X=E X+\varepsilon_{0}\right)=\frac{1-p}{2}\\
&p=P(X=E X)
\end{aligned}
$$
\end{document}

