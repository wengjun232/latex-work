
\documentclass[UTF8,a4paper,10pt]{ctexart}
\usepackage[left=2.50cm, right=2.50cm, top=2.50cm, bottom=2.50cm]{geometry}

% -- text font --
% compile using Xelatex

%\setmainfont{Microsoft YaHei}  % 微软雅黑
%\setmainfont{YouYuan}  % 幼圆    
%\setmainfont{NSimSun}  % 新宋体
%\setmainfont{KaiTi}    % 楷体
%\setmainfont{SimSun}   % 宋体
%\setmainfont{SimHei}   % 黑体

\usepackage{times}
\usepackage{diagbox}
%\usepackage{mathpazo}
%\usepackage{fourier}
%\usepackage{charter}
%\usepackage{helvet}

\usepackage{amsmath, amsfonts, amssymb} % math equations, symbols
\usepackage[english]{babel}
\usepackage{color}      % color content
\usepackage{graphicx}   % import figures
\usepackage{url}        % hyperlinks
\usepackage{bm}         % bold type for equations
\usepackage{multirow}
\usepackage{booktabs}
\usepackage{epstopdf}
\usepackage{epsfig}
\usepackage{algorithm}
\usepackage{algorithmic}
%\renewcommand{\algorithmicrequire}{ \textbf{Input:}}     % use Input in the format of Algorithm  
%\renewcommand{\algorithmicensure}{ \textbf{Initialize:}} % use Initialize in the format of Algorithm  
%\renewcommand{\algorithmicreturn}{ \textbf{Output:}}     % use Output in the format of Algorithm  

\def\deriv{d\,}
\usepackage{fancyhdr}   % 设置页眉、页脚
\pagestyle{fancy}
\lhead{2021概率论统计分析及量测技术作业}
\chead{}
\rhead{第五章:大数定律和中心极限定理}
\cfoot{\thepage}
   

\usepackage{hyperref}   % bookmarks
\hypersetup{colorlinks, bookmarks, unicode} % unicode


\title{第四章作业}
%\author{ 作者 Author \thanks{作者介绍 Brief introduction} }
%\date{\today}

\begin{document}
\section{课本习题}
\subsection{129/7}
一食品店有三种蛋糕出售,由于售出哪一种蛋糕是随机的, 因而售出一只蛋糕的价格是一个随机变量,它取 1 元、 $1.2$ 元 $、 1.5$ 元各个值的概率分别为 $0.3,0,2,0.5$. 若售出 300 只蛋糕.


(1) 求收入至少 400 元的概率.


(2) 求售出价格为 $1.2$ 元的蛋糕多于 60 只的概率.


\subsection{129/9}
 已知在某十字路口,一周事故发生数的数学期望为 $2.2$, 标准差为 $1.4$.
 
 
(1) 以 $\bar{X}$ 表示一年(以 52 周计)此十字路口事故发生数的算术平均,试用中心极限定理 求 $\bar{X}$ 的近似分布, 并求 $P\{\bar{X}<2\}$.


(2) 求一年事故发生数小于 100 的概率.

\subsection{129/11}
 随机地选取两组学生, 每组 80 人, 分别在两个实验室里测量某种化合物的 $\mathrm{pH}$. 各人 测量的结果是随机变量, 它们相互独立,服从同一分布, 数学期望为 5 ,方差为 $0.3$, 以 $\bar{X}, \bar{Y}$ 分 别表示第一组和第二组所得结果的算术平均.
 
 
(1)求$P(4.9<\bar{X}<5.1)$


(2)求$P(-0.1<\bar{X}-\bar{Y}<0.1)$

\section{补充题}
\subsection{}
现有一批建筑房屋用的木材, 其中 $80 \%$ 的长度不小于 $3 \mathrm{~m}$ 。现从这批木材中 随机抽取 100 根 试用如下两种方法分别计算其中短于 $3 \mathrm{~m}$ 的木材数在 $15 \sim 25$ 的概率:

(1)用切比雪夫不等式估计;


(2)用中心极限定理计算。

\subsection{}
设 $\left(X_{1}, X_{2}, \cdots, X_{n}\right)$ 为来自泊松分布 $\pi(\lambda)$ 的一个样本, $\bar{X}, S^{2}$ 分别为样本均 值和方差, 求 $\boldsymbol{E}(\overline{\boldsymbol{X}}), \boldsymbol{D}(\overline{\boldsymbol{X}}) 、 \boldsymbol{E}\left(\boldsymbol{S}^{2}\right)$ 。
\end{document}
